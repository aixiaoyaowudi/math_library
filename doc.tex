\documentclass[8pt, a4paper, oneside, twocolumn]{article}
\usepackage{amsmath, amsthm, amssymb, bm, graphicx, hyperref, mathrsfs, listings, fontspec, color, geometry}
\geometry{left=0.5cm, right=0.5cm, top=0.5cm, bottom=0.5cm}
\setmainfont{Times New Roman}

\definecolor{background}{RGB}{39, 40, 34}
\definecolor{string}{RGB}{230, 219, 116}
\definecolor{comment}{RGB}{117, 113, 94}
\definecolor{normal}{RGB}{248, 248, 242}
\definecolor{identifier}{RGB}{166, 226, 46}

\lstset{
	showspaces=false,
	showstringspaces=false,
	showtabs=false,
	tabsize=4,
	captionpos=b,
	breaklines=true,
	breakatwhitespace=true,
	basicstyle=\scriptsize\fontspec{Fira Code}\color{normal}\ttfamily,
	keywordstyle=\color{magenta}\ttfamily,
	stringstyle=\color{string}\ttfamily,
	commentstyle=\color{comment}\ttfamily,
	emph={format_string, eff_ana_bf, permute, eff_ana_btr},
	emphstyle=\color{identifier}\ttfamily,
	backgroundcolor=\color{background}
}

\title{{\Huge{\textbf{Project Euler Math Library}}}\\convenient \& efficient}
\author{xiaoyaowudi}
\date{\today}
\linespread{1.1}
\begin{document}
\maketitle
\pagenumbering{arabic}
\setcounter{page}{1}
\tableofcontents
\part{Modulo Operations}
$\mathrm{Modulus}$ refers to the global modulo number.
\section{Mod Int}
\subsection{montgomery\_int\_lib}
\noindent \large \textbf{Definition} \normalsize
\begin{lstlisting}[language=c++]
struct montgomery_int_lib;
\end{lstlisting}
\noindent \large \textbf{Description} \normalsize \\
Is used to store values needed for \href{https://en.wikipedia.org/wiki/Montgomery_modular_multiplication}{Montgomery Reduction}.\\
\noindent \large \textbf{Members} \normalsize \\
\textbf{P}:\\
\indent\textbf{type}: unsigned int\\
\indent\textbf{brief}: $\mathrm{Modulus}$\\
\textbf{P2}:\\
\indent\textbf{type}: unsigned int\\
\indent\textbf{brief}: $2\times \mathrm{Modulus}$\\
\textbf{NP}:\\
\indent\textbf{type}: unsigned int\\
\indent\textbf{brief}: $R^2\bmod\mathrm{Modulus}$\\
\textbf{Pk}:\\
\indent\textbf{type}: unsigned int\\
\indent\textbf{brief}: $R^{-1}\bmod\mathrm{Modulus}$\\
\textbf{redu}:\\
\indent\textbf{type}: function\\
\indent\textbf{params}: k(\textbf{type}: unsigned long long)\\
\indent\textbf{return}: $k\cdot R^{-1}\bmod \left(2\times\mathrm{Modulus}\right)$(\textbf{type}: unsigned int)\\
\textbf{reds}:\\
\indent\textbf{type}: function\\
\indent\textbf{params}: k(\textbf{type}: unsigned int, $k\in \left[0,2\times \mathrm{Modulus}\right)$)\\
\indent\textbf{return}: $k\bmod \mathrm{Modulus}$(\textbf{type}: unsigned int)\\
\textbf{redd}:\\
\indent\textbf{type}: function\\
\indent\textbf{params}: k(\textbf{type}: unsigned int, $k\in \left[0,4\times \mathrm{Modulus}\right)$)\\
\indent\textbf{return}: $k\bmod \left(2\times\mathrm{Modulus}\right)$(\textbf{type}: unsigned int)\\
\textbf{montgomery\_int\_lib}\\
\indent\textbf{type}: function\\
\indent\textbf{params}: P(\textbf{type}: unsigned int, $\mathrm{Modulus}$)\\
\indent\textbf{breif}: constructor with given Modulus.\\
\textbf{montgomery\_int\_lib}\\
\indent\textbf{type}: function\\
\indent\textbf{breif}: default constructor.\\
\subsection{montgomery\_int}
\noindent \large \textbf{Definition} \normalsize
\begin{lstlisting}[language=c++]
struct montgomery_int;
\end{lstlisting}
An implemention of \href{https://en.wikipedia.org/wiki/Montgomery_modular_multiplication}{Montgomery Reduction}.This struct has the same size as PV and has implemented add,minus,and multiply.Use 'mint' instead of this in the program.\\
\textbf{operator}:\\
\indent\textbf{type}: function\\
\indent\textbf{brief}: *,*=,+,+=,-,-=,=,-(negative),<<(io),>>(io).\\
\textbf{montgomery\_int}\\
\indent\textbf{type}: function\\
\indent\textbf{brief}: default constructor, set val to $0$.\\
\textbf{montgomery\_int}\\
\indent\textbf{type}: function\\
\indent\textbf{params}: a(\textbf{type}: const montgomery\_int \&)\\
\indent\textbf{brief}: copy constructor.\\
\textbf{montgomery\_int}\\
\indent\textbf{type}: function\\
\indent\textbf{params}: v(\textbf{type}: unsigned int)\\
\indent\textbf{brief}: initialize with value v.\\
\textbf{real\_val}\\
\indent\textbf{type}: function\\
\indent\textbf{return}: real value(\textbf{type}: unsigned int)\\
\textbf{mlib}\\
\indent\textbf{type}: static montgomery\_int\_lib\\
\indent\textbf{brief}: library, is threadprivate by default if openmp is used.\\
\subsection{montgomery\_mm256\_lib}
\noindent \large \textbf{Definition} \normalsize
\begin{lstlisting}[language=c++]
struct montgomery_mm256_lib;
\end{lstlisting}
\noindent \large \textbf{Description} \normalsize \\
Is used to store values needed for \_\_m256i \href{https://en.wikipedia.org/wiki/Montgomery_modular_multiplication}{Montgomery Reduction}.\\
\noindent \large \textbf{Members} \normalsize \\
\textbf{P}:\\
\indent\textbf{type}: \_\_m256i $8\times 32$-bit int\\
\indent\textbf{brief}: $\mathrm{Modulus}$\\
\textbf{P2}:\\
\indent\textbf{type}: \_\_m256i $8\times 32$-bit int\\
\indent\textbf{brief}: $2\times \mathrm{Modulus}$\\
\textbf{NP}:\\
\indent\textbf{type}: \_\_m256i $8\times 32$-bit int\\
\indent\textbf{brief}: $R^2\bmod\mathrm{Modulus}$\\
\textbf{Pk}:\\
\indent\textbf{type}: \_\_m256i $8\times 32$-bit int\\
\indent\textbf{brief}: $R^{-1}\bmod\mathrm{Modulus}$\\
\textbf{redu}:\\
\indent\textbf{type}: function\\
\indent\textbf{params}: k(\textbf{type}:\_\_m256i $4\times 64$-bit int)\\
\indent\textbf{return}: $k\cdot R^{-1}\bmod \left(2\times\mathrm{Modulus}\right)$(\textbf{type}: \_\_m256i $4\times 32$-bit int)\\
\textbf{reds}:\\
\indent\textbf{type}: function\\
\indent\textbf{params}: k(\textbf{type}: \_\_m256i $8\times 32$-bit int, $k\in \left[0,2\times \mathrm{Modulus}\right)$)\\
\indent\textbf{return}: $k\bmod \mathrm{Modulus}$(\textbf{type}: \_\_m256i $8\times 32$-bit int)\\
\textbf{redd}:\\
\indent\textbf{type}: function\\
\indent\textbf{params}: k(\textbf{type}: \_\_m256i $8\times 32$-bit int, $k\in \left[0,4\times \mathrm{Modulus}\right)$)\\
\indent\textbf{return}: $k\bmod \left(2\times\mathrm{Modulus}\right)$(\textbf{type}: \_\_m256i $8\times 32$-bit int)\\
\textbf{mul}:\\
\indent\textbf{type}: function\\
\indent\textbf{params}: $k_1$(\textbf{type}: \_\_m256i $8\times 32$-bit int, $k_1\in \left[0,2\times \mathrm{Modulus}\right)$), $k_2$(\textbf{type}: \_\_m256i $8\times 32$-bit int, $k_2\in \left[0,2\times \mathrm{Modulus}\right)$)\\
\indent\textbf{return}: $k_1\times k_2\bmod \left(2\times\mathrm{Modulus}\right)$(\textbf{type}: \_\_m256i $8\times 32$-bit int)\\
\textbf{add}:\\
\indent\textbf{type}: function\\
\indent\textbf{params}: $k_1$(\textbf{type}: \_\_m256i $8\times 32$-bit int, $k_1\in \left[0,2\times \mathrm{Modulus}\right)$), $k_2$(\textbf{type}: \_\_m256i $8\times 32$-bit int, $k_2\in \left[0,2\times \mathrm{Modulus}\right)$)\\
\indent\textbf{return}: $k_1+ k_2\bmod \left(2\times\mathrm{Modulus}\right)$(\textbf{type}: \_\_m256i $8\times 32$-bit int)\\
\textbf{sub}:\\
\indent\textbf{type}: function\\
\indent\textbf{params}: $k_1$(\textbf{type}: \_\_m256i $8\times 32$-bit int, $k_1\in \left[0,2\times \mathrm{Modulus}\right)$), $k_2$(\textbf{type}: \_\_m256i $8\times 32$-bit int, $k_2\in \left[0,2\times \mathrm{Modulus}\right)$)\\
\indent\textbf{return}: $k_1- k_2\bmod \left(2\times\mathrm{Modulus}\right)$(\textbf{type}: \_\_m256i $8\times 32$-bit int)\\
\textbf{montgomery\_int\_lib}\\
\indent\textbf{type}: function\\
\indent\textbf{params}: P(\textbf{type}: unsigned int, $\mathrm{Modulus}$)\\
\indent\textbf{breif}: constructor with given Modulus.\\
\textbf{montgomery\_int\_lib}\\
\indent\textbf{type}: function\\
\indent\textbf{breif}: default constructor.\\
\subsection{montgomery\_mm256\_int}
\noindent \large \textbf{Definition} \normalsize
\begin{lstlisting}[language=c++]
struct montgomery_mm256_int;
\end{lstlisting}
An implemention of \_\_m256i \href{https://en.wikipedia.org/wiki/Montgomery_modular_multiplication}{Montgomery Reduction}.This struct has the same size as PV and has implemented add,minus,and multiply.Use 'm256int' instead of this in the program.\\
\textbf{operator}:\\
\indent\textbf{type}: function\\
\indent\textbf{brief}: *,*=,+,+=,-,-=,=.\\
\textbf{montgomery\_mm256\_int}\\
\indent\textbf{type}: function\\
\indent\textbf{brief}: default constructor, set val to $0$.\\
\textbf{montgomery\_mm256\_int}\\
\indent\textbf{type}: function\\
\indent\textbf{params}: a(\textbf{type}: montgomery\_mm256\_int \&)\\
\indent\textbf{brief}: copy constructor.\\
\textbf{montgomery\_mm256\_int}\\
\indent\textbf{type}: function\\
\indent\textbf{params}: v(\textbf{type}: \_\_m256i)\\
\indent\textbf{brief}: initialize with value v.\\
\textbf{real\_val}\\
\indent\textbf{type}: function\\
\indent\textbf{return}: real value(\textbf{type}: \_\_m256i)\\
\textbf{mlib}\\
\indent\textbf{type}: static montgomery\_mm256\_lib\\
\indent\textbf{brief}: library, is threadprivate by default if openmp is used.\\
\subsection{Others}
\textbf{global\_mod}\\
\indent\textbf{type}: unsigned int\\
\indent\textbf{brief}: refers to the global Modulus in current thread, is threadprivate by default if openmp is used.\\
\textbf{set\_mod}\\
\indent\textbf{type}: function\\
\indent\textbf{params}: P(\textbf{type}: unsigned int, $\mathrm{Modulus}$)\\
\indent\textbf{breif}: set the global Modulus in the current thread to $P$.\\
\textbf{set\_mod\_for\_all\_threads}\\
\indent\textbf{type}: function\\
\indent\textbf{params}: P(\textbf{type}: unsigned int, $\mathrm{Modulus}$)\\
\indent\textbf{breif}: set the global Modulus in the all threads to $P$ if openmp is used, otherwise is equivalent to set\_mod.\\
\end{document}